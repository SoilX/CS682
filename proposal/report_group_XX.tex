\documentclass[twocolumn]{article}

\usepackage{amsmath}
\usepackage{graphicx}
\usepackage{xcolor}
\usepackage[utf8]{inputenc}

\newcommand{\todo}[1]{\textcolor{red}{#1}}

\title{CS482/682 Final Project Report Group \todo{XX}\\
  \large Using Unsupervsed Learned to provide a Competive Edge in a Classification
Competition}

\author{Tianyu Ding, Ivan Liao, Dylan Madisetti, Alex Sun}

\date{}

\begin{document}

  \maketitle

  % do not write an abstract
  \section{Introduction}

  As a case study in how unsupervised learning can be used to tackle
  established, benchmarkable problems, this project seeks to compete in the
  \href{https://www.kaggle.com/c/pku-autonomous-driving}{Peking University/Baidu
  Autonomous Vehicle Challenge}. This challenge provides pictures of traffic
  while driving, and requires a prediction of the captured cars model type, and
  positioning (yaw, pitch, roll, x, y, z). Auxiliary data is provided in the
  form of 3D models for the cars of interest, and segmentation masks of the cars
  against the background.

  An unsupervised approach to solving this problem is proposed by a 3 stage
  progressive pipeline. Each stage is a standalone, unsupervised methodology and
  thus allows for incremental progress in tackling this challenge. It is
  understood that the competition task is non-trivial- thus this approach will
  guarantee some concrete deliverables.

  \textbf{Stage 1}. In the first stage, unsupervised segmentation of . This
  methodology can be benchmarked against the provided segmentation masks.
  Completion of this stage will allow for car segmentation from an arbitrary
  image. Further work may include benchmarking this unsupervised approach to a
  supervised approach (using the provided segmentation masks as ground truth).

  \textbf{Stage 2}. In the second stage, an embedding space of masked cars will
  be produced. Latent embedding spaces, contain powerful unsupervised
  information. Here, the provided masks will be used to extract cars from the
  background. A convolutional neural net, followed by dense layers will map this
  segmentation into some latent space. This latent space will then be "grown"
  into a representation as with standard auto-encoders. To ensure task relevant
  information, batches with unique car labels will be used. An added loss task
  of weakly enforcing orthogonality will then be prescribed. Given a logit
  representation "E" in the embedding space, this orthogonality constraint can be
  expressed as:

  $$\text{Loss}_\text{extra} = \alpha ||\frac{E\cdot E^T}{\epsilon |E|} - \mathbb{I}||_{L_p}, 1 > p > \infinty$$
  
  For some measure $L_p$, to be determined experimentally, and some
  hyper-parameter $\alpha$, that dictates the influence of the added task. We
  note here that more overlap that an embedding has with another will produce a
  larger loss. The net loss will thus be given by:

  $$\text{Loss} = \text{Loss}_\text{autoencoder} + \text{Loss}_\text{extra}$$

  This space will then be clustered to provide transparency into the structure
  of the latent space.  Specification and hyperparamters for these layers are to be determined-
  however, our teams make special notice of the following networks that have had
  success in processing image data: \todo{something \ref{}} . Note that this stage can occur
  independently of Stage 1. Completion of this task should be a fully functional
  auto-encoder, and a beautiful (hopefully) map of the latent space using a
  method like
  \href{tsne}{http://www.jmlr.org/papers/volume9/vandermaaten08a/vandermaaten08a.pdf}.

  \textbf{Stage 3} The final aspect of the pipeline produces a labeled embedding
  space.  
  The segmentation vectors in the space produced by Stage 2, will be mapped will
  be mapped to a new space shared with learned model labels. This
  idea is borrowed from \href{https://arxiv.org/pdf/1705.00652.pdf}{Smart Reply} and can be bested expressed as semi-supervised
  learning. This methodology is chosen such that an unspecified number of models
  can be used in classification. In a given batch (with unique labels), a
  softmax loss can be used by matching the labels to the given sample.
  Further work includes developing means of
  visualizing the latent "car space" by implementing information from the
  provided 3D models.  \todo{viz latent spaces \ref{}}.

  These pipelines will initially be trained independently on a segment of the
  training data (refered to here as validation data). Later, the pipeline pieces
  will feed each other and be training from end to end on the remaining dataset.
  Labels can be extracted from the cosine comparison of labels and a given
  encoding, and additional layers will be added to predict positioning. The
  robustness of this technique can be expanded by feeding in augmented images as
  exemplified by \todo{augmentation paper\ref{}}

  \todo{Add in diagram of pipelines}

  \todo{add in more referenes}
  \todo{todo get feedback from everyone}

  \paragraph{Background}
  \paragraph{Related Work} \todo{e.g. previous supervised approaches, other unsupervised methods etc}

  \section{Methods}
  \paragraph{Dataset} \todo{BRIEFLY describe the dataset and any pre-processing you may use that is special (e.g. downsample all images by x8 to enable colaboratory training etc.)}

  \paragraph{Setup, Training and Evaluation} \todo{IMPORTANT: you can change this paragraph to better fit your project. Questions to answer: What architecture did you chose and why? How did you accomplish un- or self-supervised learning? What tweaks were necessary to make this happen, e.g. custom layer designs, auxiliary tasks, etc.?}



  \section{Results}


  \section{Discussion} \todo{make sure you use this section to showcase your understanding of the results you achieved and whether you could have / should have done things differently }

\end{document}
